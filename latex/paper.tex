\documentclass{article}
\usepackage{hyperref}
\usepackage{float}
\usepackage{url}
% Title Information
\title{Values Encoded in Machine Learning Research}
\author{Isabel Kurth\thanks{Hasso Plattner Institute, Potsdam, Germany.\ \texttt{isabel.kurth@student.hpi.de}}}

% Document Start
\begin{document}

\maketitle
\thispagestyle{plain}

% Abstract Section
\begin{abstract}
% \textbf{Keywords:} Machine Learning, Ethics, Values, Society, Impact
\end{abstract}

% Sections
\section{Introduction}
Machine learning (ML) research has become a pivotal driver of technological advancement, shaping fields as diverse as healthcare, finance, 
and autonomous systems. Despite its transformative potential, the values embedded in influential ML research remain underexamined, raising 
critical questions about whom the field serves and whose needs it prioritizes. Birhane et al. (2023)~\cite{valuesInML2021} underscore that ML research often prioritizes 
performance, generalization, and novelty while neglecting societal needs, ethical considerations, and broader societal impact.

Inspired by this work, our study systematically investigates the values reflected in the 100 most influential papers from NeurIPS 2023 and ICML 2023. 
These conferences represent the cutting edge of ML research and are key arenas where emergent trends and priorities are crystallized. 
By analyzing these highly-cited papers, we aim to uncover patterns in their value commitments, especially regarding how they balance technical rigor 
with societal impact. This work aims to investigate if the most encoded values changed from the years 2008/09 and 2018/19 
(investigated by Birhane et al. (2023)) to more recent papers from 2023.

This paper contributes to the growing discourse on the socio-political dimensions of ML research by providing empirical insights into the values 
upheld in contemporary work. Through our analysis, we aim to encourage researchers, institutions, and policymakers to critically reflect on the 
long-term implications of prioritizing certain values over others.

\section{Methodology}
\subsection{Data collection}
We curated a dataset of the 100 most influential papers from NeurIPS 2023 and ICML 2023 using rankings provided by Paper Digest 
(\url{https://www.paperdigest.org/topic/?topic=nips&year=2023} and \url{https://www.paperdigest.org/topic/?topic=icml&year=2023}). 
These rankings were based on an impact score, which reflects a combination of paper citations, patent citations, etc. It is a score in 1.0-10.0, 
with a higher value indicating a broader impact. We then downloaded the respective papers from the conference website directly 
(\url{https://proceedings.neurips.cc/paper_files/paper/2023} and \url{https://icml.cc/Downloads/2023}). 
The selected papers represent the cutting edge of machine 
learning research and provide a representative sample of current trends and values in the field. We did not use Semantic Scholar as in the Birhane et 
al. (2023) paper because Semantic Scholar did not provide API keys due to high demand. We could also not use Google Scholar to access the citation 
counts because the IP address gets blocked after too many access tries. A list of the PDFs of the 100 papers of both conferences can be found in the 
GitHub repository.

\subsection{Keyword based analysis}
To analyze the values encoded in these papers, we adopted the 74 keywords used by Birhane et al. (2023) in their study of influential ML papers. In their paper
they write that they use 67 values, but in their annotation template for manual annotations (annotations.tsv) they include 74 values. In their results folder they
just mention 62 values. In order to search for as many values as possible we decided to proceed with the 74 values from the annotation file in their data folder.  
These keywords include terms related to performance, generalization, novelty, societal impact, and ethical considerations. 
Unlike Birhane et al., who relied on manual annotation, we developed and deployed an automated keyword-scanning algorithm to systematically identify the presence of 
these keywords across the abstracts, introductions, and conclusions of the selected papers.
The Table\ref{tab:keyword_table} in the Appendix provides a comprehensive list of the values and their associated keywords used in our analysis.
For example we did not only query for the additonal keyword ``Novelty'' but also for ``novel'' as these two
words express the same value and because we use a keyword search, sentences with the word ``novel'' would not be counted towards the keyword ``Novelty''.  
For each keyword occurrence, contextual information was extracted to ensure accurate interpretation. 
This semi-automated process balanced the scale of automated analysis with the depth of manual review. One major limitation to this keyword based approach is 
that for some keywords this works better than for others and these keywords might then be under- or overrepresented in the analysis. For example for the keyword ``Fairness'' is 
is really likeli that the keyword search will find the occurences if the paper adresses this topic because they will most liekli use the word in the context. 
As a counter example the keyword `Èasy to implement'' might not be found as easily because the word ``easy'' is used in many other contexts as well and therefore we can not query for it.
There are many different words in witch the researchers could express that their model is easy to implement and they are hard to capture all by a simple keyword search. In order to keep that limitations as simple as possible
we try to query for many similar keywords and hope that this will cover most of the values.
Even tough the keyword search is not perfect, it is a good way to get a first overview of the values encoded in the papers and to get a first impression of the values. To stay in the scope of this seminar paper 
we therefor decided to use this approach but we want to explicitly mention this limitation.

\subsection{Quantitative Summary}
We measured the prevalence of each keyword in the dataset, calculating the relative frequency of each value. In Figure 1, we present the distribution of the values across all 200 papers from both conferences. The top values 
are Performance, Quantitative evidence, Qualitative evidence and Generality. We specifically highlited the values that the Brihane et al. paper groups as ``User rights'' and ``Ethical Principles'' in the figure.

\subsection{Qualitative Summary}
Excerpts containing the keywords were manually reviewed to explore how values are framed and justified. Particular attention was given to the framing of societal impacts and ethical considerations, drawing comparisons with prior findings by Birhane et al.

% Table Example
% \begin{table}[h]
%     \centering
%     \begin{tabular}{lllrrr}
%         \toprule
%         Testing Data & Sources & Hardness & Rule type-1 & Rule type-2 & Rule type-3\\
%         \midrule
%         Sinica & Balanced corpus & Moderate & 92.97 & 94.84 & 96.25 \\
%         Sinorama & Magazine & Difficult & 90.01 & 91.65 & 93.91\\
%         Textbook & Elementary school & Easy & 93.65 & 95.64 & 96.81 \\
%         \bottomrule
%     \end{tabular}
%     \caption{The 50-best oracle performances from the different grammars.}
% \end{table}

% References
\bibliographystyle{plain}
\bibliography{references}

\appendix
\section{Keyword Analysis Table}

%%%%% muss geupdated werden 
\begin{table}[h]
    \centering
    \begin{tabular}{|l|l|}
        \hline
        \textbf{Keywords Birhane Paper} & \textbf{additional keywords for automated keyword search} \\
        \hline
        Novelty & novel \\
        Simplicity & simple \\
        Generalization & generalisable \\
        Flexibility/Extensibility & flexible, flexibility, extensibility, extensible \\
        Robustness & robust \\
        Realistic output & realistic \\
        Formal description/analysis & formal, mathematical \\
        Theoretical guarantees & guarantee \\
        Approximation & approximate \\
        Quantitative evidence (e.g. experiments) & quantitative \\
        Qualitative evidence (e.g. examples) & qualitative \\
        Scientific methodology & scientific \\
        Controllability (of model owner) & control \\
        Human-like mechanism & human \\
        Low cost & cheap, cost \\
        Large scale & scale \\
        Generality & general \\
        Principled & principles \\
        Exactness & exact \\
        Preciseness & precise \\
        Concreteness & correct \\
        Automatic & automated \\
        Efficiency & efficient \\
        Building on classic work & classic work \\
        Building in recent work & recent work \\
        Unifying ideas or integrating components & unifying \\
        Identifying limitations & limitations \\
        Critique & criticism \\
        Understanding (for researchers) & understanding \\
        Used in practice/Popular & practice, popular \\
        Reproducibility & reproduce \\
        Easy to implement & implement \\
        Requires few resources & resources \\
        Parallelizability / distributed & parallelizability, parallelization, distributed \\
        Facilitating use (e.g. sharing code) & sharing code \\
        Scales up & scale up \\
        Applies to real world & real world \\
        Learning from humans & humans, learning \\
        Practical & practice \\
        Useful & usefulness \\
        Interpretable (to users) & interpretable \\
        Transparent (to users) & transparent, transparency \\
        Privacy & privacy, private \\
        Fairness & fair \\
        Not socially biased & social bias, socially bias, social, society \\
        User influence \\
        Collective influence & collective \\
        Deferral to humans & deferral \\
        Critiqability & criticism \\
        Beneficence & beneficable \\
        Respect for Persons & respect \\
        Autonomy (power to decide) & autonomy, autonome \\
        Explicability & explicable \\
        Respect for Law and public interest & respect for law, respect for public interest \\
        Security & secure \\
        Easy to work with & easy \\
        Realistic world model & world model \\
        Fast & speed \\
        \hline
    \end{tabular}
    \caption{Values and their associated keywords used in the analysis.}
    \label{tab:keyword_table}
\end{table}



\end{document}
